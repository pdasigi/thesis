Since human language is aimed at other humans who share 
the same background knowledge, a lot of information is often not explicitly stated
for the sake of brevity. The implicit knowledge required to understand language varies 
from simple commonsense in the case of basic conversations, to complex principles that link 
concepts in more esoteric communications. Consider an example of the former in solving the 
textual entailment problem given the following premise, and a candidate hypothesis:
\begin{itemize}
 \item \textbf{Premise} \textit{Children and parents are splashing water at the pool.}
 \item \textbf{Hypothesis} \textit{Families are playing outside.}
\end{itemize}
The knowledge a human uses to correctly predict that the hypothesis can be inferred from the 
premise is that \textit{children and parents} typically form \textit{families}, \textit{splashing water} 
is a kind of \textit{playing}, and a \textit{pool} is expected to be \textit{outside}. Similarly, an 
automated system that reads a science text book to answer the following 
question requires the knowledge of properties of materials.
\begin{itemize}
 \item \textit{Which of the following is the best conductor of electricity?} 
 A. \textit{glass rod} B. \textit{wooden stick} C. \textit{plastic straw} D. \textit{metal nail}
\end{itemize}

%TODO: Better definition of NLU
Natural Language Understanding (NLU) systems process human generated text 
(or speech) at a deep semantic level and output a complete meaning representation of the input. 
Recent advances in NLU rely heavily on representation learning or deep learning
approaches that model 
text meaning in context. While these methods have been successfully used to represent meaning
of words in the context of other words, and thus learn complex functions of their distributional
properties, the other aspects of semantics, like those needed in the examples above are out of their reach.
% What are the other aspects of semantics? And why do distributional models not capture them?
%TODO: Propose a generic formulation of the problem
%TODO: Lead to the need for external inputs
Fortunately, the additional knowledge humans would use to do deep semantic processing is encoded, to
some extent in knowledge bases (KB) and ontologies. However, using that information in NLU systems is a 
hard task. Firstly, while linking the words being read to the structured knowledge
in a KB, an automated system faces ambiguity. For example, with lexical ontologies like WordNet, 
we get useful type hierarchies like \textit{parent is-a ancestor is-a person} and \textit{pool is-a body-of-water} 
and so on, but one has to deal with sense ambiguity: \textit{pool} can also be a game. Moreover, finding the 
relevant parts of the KB given the text being processed is a challenge. For example, assuming we have a KB that encodes
the properties of materials, to answer the question in the second example above, we still have to 
find the conducting properties, and possibly generalize the properties of various specific metals to
infer that metals are generally good conductors of electricity.

%Modify the generic formulation to incorporate knowledge

The goal of this thesis is to find the limitations of distributional and the ontological 
sources of information and use this knowledge to build hybrid NLU systems that can successfully 
incorporate structured background knowledge in deep learning models. The fact
that the two sources of semantic information are fundamentally different 
makes this challenging. While distributional approaches encode meaning in a
continuous and an abstract fashion, meaning in KBs is symbolic and discrete. 
I will dedicate a major chunk of this thesis to methods of incorporating
symbolic knowledge of various kinds in neural networks, and learning 
distributions over the discrete concepts of the KB conditioned on the context,
to deal with exceptions in language.


% TODO: Show a generic deep learning pipeline for NLU and talk about which parts can be improved and how.

\section{Outline}
In the first part of this thesis, I will explore the limits of purely distributional models
and study the kinds of information that cannot be modeled by them. A candidate problem for this
exercise is semantic anomaly detection, which involves automatically 
identifying real world commonsense violations in text. I shall describe an annotation
effort that resulted in newswire headlines manually labeled with the degree
of surprise associated with them. The task is to build a model that can identify the highly
surprising headlines as anomalies. This is different from the usual Language Modeling (LM) problem
because even the semantic anomalies are well-formed sentences that can be well-understood. Consequently,
the model needs to discriminate between sentences at a semantic level deeper than surface fluency.
I shall explore to what extent neural network language models (NNLM), the current state of the art in LM,
are applicable in semantic anomaly detection. A popular technique used to avoid normalizing the probabilities
over the entire vocabulary in NNLM is Noise Contrastive Estimation (NCE), which modifies the problem of 
building a generative model into that of building a discriminative one to distinguish the data distribution
from a predefined noise distribution. A challenge in building NNLM for semantic anomaly detection is that defining 
the noise distribution in this case is not trivial. One way to address this challenge is to use ideas from 
adversarial networks, where the noise samples themselves are generated by a neural network whose complexity
is comparable to that of the network that discriminates the noise from the data. A baseline model for this 
problem is a previously published supervised Recursive Neural Network trained for the same problem.


Next, I will describe a method to incorporate selectional 
preferences in Machine Reading. This involves a variant of Long Short-Term
Memory (LSTM) based Recurrent Neural Networks (RNN) that use information
from WordNet, a lexical 
ontology. The hybrid model looks at WordNet synsets, and the hypernym hierarchies of the
words being processed so that the 
LSTM is aware of their different senses, and the corresponding type information.
The ontology aware LSTM (OntoLSTM) learns to attend to the appropriate sense,
and the relevant type 
(either the most specific concept or a generalization by choosing a hypernym of
the word), conditioned on the context and the end-objective. I show that when
trained in an 
unsupervised setting as a Language Model (LM), OntoLSTM has a lower perplexity
than traditional LSTMs, and also does well at unsupervised Word Sense
Disambiguation (WSD). In a supervised setting, 
it outperforms traditional LSTMs in predicting textual entailment.

The third part of the thesis contains extensions of the idea to model factual knowledge, towards
systems that answer non-factoid questions, particularly in general science. Such systems 
need to reason about general principles behind facts, and thus distributional information
alone is not enough. I will show methods that encode graph based features from KBs relevant to the 
question, as high dimensional vector representations such that they can provide 
additional context. Whereas the structured information used in the first part has only one kind of
relation, the KB used in this part has several relation types. 

The previous two parts of the thesis deal with words or multi-word expressions as units being grounded in KBs. 
There are other kinds of frame structures that these techniques can be extended to. One example is the rhetorical 
structure in discourse. The units of operation in this case are clauses instead of words. This is one potential 
direction for the third part of the thesis. Another direction is automatic construction of lexical ontologies in 
low resource languages. Applying the techniques described in the second part, we may be able to induce senses and 
hierarchies of concepts in a new language using only distributional information.
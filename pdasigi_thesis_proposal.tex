%for a more compact document, add the option openany to avoid
%starting all chapters on odd numbered pages
\documentclass[12pt]{cmuthesis}

% This is a template for a CMU thesis.  It is 18 pages without any content :-)
% The source for this is pulled from a variety of sources and people.
% Here's a partial list of people who may or may have not contributed:
%
%        bnoble   = Brian Noble
%        caruana  = Rich Caruana
%        colohan  = Chris Colohan
%        jab      = Justin Boyan
%        josullvn = Joseph O'Sullivan
%        jrs      = Jonathan Shewchuk
%        kosak    = Corey Kosak
%        mjz      = Matt Zekauskas (mattz@cs)
%        pdinda   = Peter Dinda
%        pfr      = Patrick Riley
%        dkoes = David Koes (me)

% My main contribution is putting everything into a single class files and small
% template since I prefer this to some complicated sprawling directory tree with
% makefiles.

% some useful packages
\usepackage{times}
\usepackage{fullpage}
\usepackage{graphicx}
\usepackage{amsmath}
\usepackage{amsfonts}
\usepackage{amssymb}
\usepackage{bbm}
\usepackage{multirow}
\usepackage{enumerate}
\usepackage[numbers,sort]{natbib}
\usepackage[backref,pageanchor=true,plainpages=false, pdfpagelabels, bookmarks,bookmarksnumbered,
%pdfborder=0 0 0,  %removes outlines around hyper links in online display
]{hyperref}
\usepackage{subfigure}

\DeclareMathOperator*{\argmin}{arg\,min}

% Approximately 1" margins, more space on binding side
%\usepackage[letterpaper,twoside,vscale=.8,hscale=.75,nomarginpar]{geometry}
%for general printing (not binding)
\usepackage[letterpaper,twoside,vscale=.8,hscale=.75,nomarginpar,hmarginratio=1:1]{geometry}

% Provides a draft mark at the top of the document. 
\draftstamp{\today}{DRAFT}

\begin {document} 
\frontmatter

%initialize page style, so contents come out right (see bot) -mjz
\pagestyle{empty}

\title{ {\it \huge Thesis Proposal}\\
{\bf Knowledge-aware Natural Language Understanding}}
\author{Pradeep Dasigi}
\date{}
\Year{2016}
\trnumber{}

\committee{
Eduard Hovy, CMU (Chair) \\
Chris Dyer, CMU, Google Deepmind \\
William Cohen, CMU \\
Sumit Chopra, Facebook AI Research
}

\support{}
\disclaimer{}

% copyright notice generated automatically from Year and author.
% permission added if \permission{} given.

\keywords{neural networks, knowledge base, hybrid NLU models}

\maketitle

% TODO: Uncomment this for the dissertation
% \begin{dedication}
% 
% \end{dedication}

\pagestyle{plain} % for toc, was empty

%% Obviously, it's probably a good idea to break the various sections of your thesis
%% into different files and input them into this file...

\begin{abstract}
Natural Language Understanding (NLU) systems
process human generated text (or speech) at a deep semantic level and organize
the information in a way that is useful for downstream applications such as
textual entailment, summarization and question answering. Recent
advances in NLU rely heavily on representation learning or deep learning approaches
that model text meaning in context. These approaches generally involve two main steps.
The first is encoding, where words (or other basic linguistic units) within the input
sentences are composed to get a unified representation per sentence. These encoded representations
are then used in the second step as features in a classifier or a structured prediction component to produce the desired
output. In this thesis, we identify two main weaknesses with this generic NLU pipeline, and propose address them by incorporating
external knowledge inputs.

Firstly, while distributional methods for 
encoding inputs have been successfully used to represent meaning of words in the context of other words,
and thus learn complex functions of their distributional properties, there are other
aspects of semantics that are out of their reach. These aspects are related to
commonsense or real world information which is part of shared human knowledge but is not explicitly
present in input. We address this issue using knowledge-aware encoders that produce representations grounded in
knowledge bases like WordNet and Freebase. 

Secondly, many NLU tasks require additional contextual information beyond a single sentence. Examples of such tasks
include machine comprehension, where the system is required to read a paragraph and answer questions about it. These tasks
require models that can perform inference over long-term memory. Recent progress in memory networks addresses this requirement
to some extent. However, the memory network models published so far have been shown to be capable of only shallow reasoning. We apply
these models to the task of answering non-factoid questions from science text books, and our initial results show that there is a lot of room
for improvement. We identify the difficulty of the the problem and propose some solutions to improve these models.

Finally, we propose to investigate the transferability of learned representations from one NLU task to another. Concretely, our experiments
will answer whether similar NLU tasks can be modeled jointly, whether decreasing the complexity of the encoders makes 
them overfit less to the task they are trained on and make them more transferable, and what role external knowledge plays in transfer learning.

\end{abstract}

% TODO: Uncomment this for the dissertation
% \begin{acknowledgments}
% 
% \end{acknowledgments}



\tableofcontents
\listoffigures
\listoftables

\mainmatter

%% Double space document for easy review:
%\renewcommand{\baselinestretch}{1.66}\normalsize

% The other requirements Catherine has:
%
%  - avoid large margins.  She wants the thesis to use fewer pages, 
%    especially if it requires colour printing.
%
%  - The thesis should be formatted for double-sided printing.  This
%    means that all chapters, acknowledgements, table of contents, etc.
%    should start on odd numbered (right facing) pages.
%
%  - You need to use the department standard tech report title page.  I
%    have tried to ensure that the title page here conforms to this
%    standard.
%
%  - Use a nice serif font, such as Times Roman.  Sans serif looks bad.
%
% Other than that, just make it look good...


\chapter{Introduction}
\label{chapter:introduction}
Natural Language Understanding (NLU) systems process human generated text (or speech) at a deep semantic level and encode the meaning
of the processed inputs. Several concrete tasks have been defined within the field on Computational Linguistics (CL) to evaluate the encoded meaning representations.
In the case of Semantic Parsing, the output is a logical form that can be queried against a knowledge base of facts to compute its truth value. Textual Entailment refers to the 
problem of identifying whether the truth of some text provided as a hypothesis follows from that of another text provided as a premise. Sentiment Analysis is the process of 
automatically categorizing the opinions expressed in the inputs towards specific targets. All these tasks require encoding the semantics of the input in a way that is
at least good enough to perform well at the given task, with the inherent assumption that an improvement in the task performance correlates with an improvement in the
quality of the semantic representation or in other words, the understanding capability of the computational system.
% TODO: Improve the following statement and uncomment it.
%Each task defines a restricted set of linguistic aspects that can be taught to a computational system, and are easy enough to automatically evaluate
%on unseen examples, but are complex enough for the representation techniques to be applicable to understanding the unrestricted set.

A generic NLU system can thus be succinctly described using the following two equations.
\begin{align}
 \mathbf{e} &= \mathtt{encode}(\mathbf{I}) \label{eq:generic_encoding}\\
 \mathbf{o} &= \mathtt{predict}(\mathbf{e}) \label{eq:generic_prediction}
\end{align}
where $\mathbf{I}$ is the set of textual inputs to the NLU system. For example, $\mathbf{I}$  are single sentences in Sentiment Analysis and pairs of sentences in Textual Entailment. $\textbf{e}$ are intermediate 
encoded semantic representations of the inputs (which may or may not be task specific), and $\mathbf{o}$ are the final task specific predictions. For example, in Sentiment Analysis or Textual Entailment, 
$\mathbf{o}$ are categorical labels indicating the sentiment or entailment respectively, and in the case of Semantic Parsing, they are structured outputs or parses. In older feature-rich methods for NLU, 
Equation~\ref{eq:generic_encoding} is a mapping of the inputs to a hand designed feature space, typically containing patterns over n-grams or shallow linguistic features like part-of-speech tags. In such systems,
the modeling emphasis was more on the prediction component, and the encoding component did not involve any learning. More recent systems use representation learning techniques to also learn the parameters of the
$\mathtt{encode}$ function, and typically this is done jointly with learning the parameters of the $\mathtt{predict}$ function.

\section{External Knowledge}
An issue with the formulation of the problem in Equation~\ref{eq:generic_encoding} and Equation~\ref{eq:generic_prediction} is that they are missing external knowledge. 
Since human language is aimed at other humans who share 
the same background knowledge, a lot of information is often not explicitly stated
for the sake of brevity. The implicit knowledge required to understand language varies 
from simple commonsense in the case of basic conversations, to complex principles that link 
concepts in more esoteric communications. Consider an example of the former in solving the 
textual entailment problem given the following premise, and a candidate hypothesis:
\begin{itemize}
 \item \textbf{Premise:} \textit{Children and parents are splashing water at the pool.}
 \item \textbf{Hypothesis} \textit{Families are playing outside.}
\end{itemize}
The knowledge a human uses to correctly predict that the hypothesis can be inferred from the 
premise is that \textit{children and parents} typically form \textit{families}, \textit{splashing water} 
is a kind of \textit{playing}, and a \textit{pool} is expected to be \textit{outside}. Similarly, an 
automated system that reads a science text book to answer the following 
question requires the knowledge of properties of materials.
\begin{itemize}
 \item \textit{Which of the following is the best conductor of electricity?}\\ 
  A. \textit{glass rod}  B. \textit{wooden stick}  C. \textit{plastic straw} D. \textit{metal nail}
\end{itemize}
Clearly, we need some external knowledge as an input to our NLU systems to perform well at tasks like this.
Without this additional context, the NLU systems will merely memorize specific patterns seen in the training data
and cannot generalize to unseen test cases. Accordingly, we modify our NLU equations as follows.
\begin{align}
 \mathbf{e} &= \mathtt{encode}(\mathbf{I}, \mathbf{K}_e) \label{eq:encoding_with_knowledge}\\
 \mathbf{o} &= \mathtt{predict}(\mathbf{e}, \mathbf{K}_p) \label{eq:prediction_with_knowledge}
\end{align}
where $\mathbf{K}_e$ and $\mathbf{K}_p$ represent the knowledge required for encoding and prediction respectively. 
They serve different purposes. While $\mathbf{K}_e$ is additional knowledge used to better compose the units needed to
encode input text, $\mathbf{K}_p$ helps with reasoning. Two examples of $\mathbf{K}_e$ are hypernym trees from WordNet to
incorporate commonsense information about concepts, and subgraph features from Freebase to encode relations
between entities seen in the input text. $\textbf{K}_p$ inputs might come from 

The additional knowledge humans use to do deep semantic processing is encoded, to
some extent in knowledge bases (KB) and ontologies. However, using that information in NLU systems is a 
hard task. Firstly, while linking the words being read to the structured knowledge
in a KB, an automated system faces ambiguity. For example, with lexical ontologies like WordNet, 
we get useful type hierarchies like \textit{parent is-a ancestor is-a person} and \textit{pool is-a body-of-water} 
and so on, but one has to deal with sense ambiguity: \textit{pool} can also be a game. Moreover, finding the 
relevant parts of the KB given the text being processed is a challenge. For example, assuming we have a KB that encodes
the properties of materials, to answer the question in the second example above, we still have to 
find the conducting properties, and possibly generalize the properties of various specific metals to
infer that metals are generally good conductors of electricity.

The goal of this thesis is to find the limitations of distributional and the ontological 
sources of information and use this knowledge to build hybrid NLU systems that can successfully 
incorporate structured or unstructured background knowledge in deep learning models. The fact
that the two sources of semantic information are fundamentally different 
makes this challenging. While distributional approaches encode meaning in a
continuous and an abstract fashion, meaning in KBs is symbolic and discrete. 
A major chunk of this thesis will be dedicated to methods of incorporating
symbolic knowledge of various kinds in neural networks, and learning 
distributions over the discrete concepts of the KB conditioned on the context,
to deal with exceptions in language.

\section{Generic NLU Pipeline}

% TODO: Show a generic deep learning pipeline for NLU and talk about which parts can be improved and how.

\section{Outline}
In the first part of this thesis, I will explore the limits of purely distributional models
and study the kinds of information that cannot be modeled by them. A candidate problem for this
exercise is semantic anomaly detection, which involves automatically 
identifying real world commonsense violations in text. I shall describe an annotation
effort that resulted in newswire headlines manually labeled with the degree
of surprise associated with them. The task is to build a model that can identify the highly
surprising headlines as anomalies. This is different from the usual Language Modeling (LM) problem
because even the semantic anomalies are well-formed sentences that can be well-understood. Consequently,
the model needs to discriminate between sentences at a semantic level deeper than surface fluency.
I shall explore to what extent neural network language models (NNLM), the current state of the art in LM,
are applicable in semantic anomaly detection. A popular technique used to avoid normalizing the probabilities
over the entire vocabulary in NNLM is Noise Contrastive Estimation (NCE), which modifies the problem of 
building a generative model into that of building a discriminative one to distinguish the data distribution
from a predefined noise distribution. A challenge in building NNLM for semantic anomaly detection is that defining 
the noise distribution in this case is not trivial. One way to address this challenge is to use ideas from 
adversarial networks, where the noise samples themselves are generated by a neural network whose complexity
is comparable to that of the network that discriminates the noise from the data. A baseline model for this 
problem is a previously published supervised Recursive Neural Network trained for the same problem.


Next, I will describe a method to incorporate selectional 
preferences in Machine Reading. This involves a variant of Long Short-Term
Memory (LSTM) based Recurrent Neural Networks (RNN) that use information
from WordNet, a lexical 
ontology. The hybrid model looks at WordNet synsets, and the hypernym hierarchies of the
words being processed so that the 
LSTM is aware of their different senses, and the corresponding type information.
The ontology aware LSTM (OntoLSTM) learns to attend to the appropriate sense,
and the relevant type 
(either the most specific concept or a generalization by choosing a hypernym of
the word), conditioned on the context and the end-objective. I show that when
trained in an 
unsupervised setting as a Language Model (LM), OntoLSTM has a lower perplexity
than traditional LSTMs, and also does well at unsupervised Word Sense
Disambiguation (WSD). In a supervised setting, 
it outperforms traditional LSTMs in predicting textual entailment.

The third part of the thesis contains extensions of the idea to model factual knowledge, towards
systems that answer non-factoid questions, particularly in general science. Such systems 
need to reason about general principles behind facts, and thus distributional information
alone is not enough. I will show methods that encode graph based features from KBs relevant to the 
question, as high dimensional vector representations such that they can provide 
additional context. Whereas the structured information used in the first part has only one kind of
relation, the KB used in this part has several relation types. 

The previous two parts of the thesis deal with words or multi-word expressions as units being grounded in KBs. 
There are other kinds of frame structures that these techniques can be extended to. One example is the rhetorical 
structure in discourse. The units of operation in this case are clauses instead of words. This is one potential 
direction for the third part of the thesis. Another direction is automatic construction of lexical ontologies in 
low resource languages. Applying the techniques described in the second part, we may be able to induce senses and 
hierarchies of concepts in a new language using only distributional information.
\chapter{Learning Representations Using Knowledge}
\label{chapter:ontolstm}
In this chapter, we show two instances of encoding textual inputs in the context of external knowledge. The first model shown here is an ontology-aware recurrent neural network, an encoder that has access to the token level semantic information from an ontology (Wordnet).
Despite their empirical success, type-level word embeddings ignore the semantic ambiguity of lexical items. We address this problem by using Wordnet to find the collection of semantic concepts manifested by a word type, and represent a word type as a collection of concept embeddings. We show
how to integrate the proposed ontology-aware lexical representation with recurrent neural networks (RNNs) to model token sequences. The second model is proposed 

\section{Ontology Aware Recurrent Neural Networks}
A human can process new information in a sentence, use context to resolve ambiguities, and use commonsense to make relevant generalizations.
For instance, when Alice tells Bob \textit{``Children and parents are splashing water at the pool,''} Bob may reasonably expect that Alice means the `swimming pool' sense of the word `pool' and also infer that \textit{``Children are playing outside with adult supervision.''}
In contrast, semantic ambiguity and commonsense inference constitute major challenges for computational models of language \citep{yarowsky:94,tanaka:07,celikyilmaz:13,pasca:14}.
Computational models used for machine reading (e.g., \cite{bowman:15}) often use the same representation for all instantiations of a word type, ignoring semantic ambiguity at the lexical level.  We focus on word embeddings as an important component of many computational models of language.
Type-level word embeddings are a lexical representation which maps a word type (i.e., a surface form)\footnote{We use the term `word type' to refer to the surface form of a word, and the term `word token' to refer to a particular instantiation of the word in text.} 
to a dense vector of real numbers such that similar word types have similar embeddings. 
Despite their empirical success (e.g. \cite{socher:10}) they inherently ignore the semantic ambiguity of word types (e.g., `pool' may refer to a swimming pool or the pocket billiards sport)  and do not explicitly model lexical generalizations (e.g., a swimming pool is a container).

The model proposed here uses WordNet \citep{miller1995wordnet} to encode generalizations and semantic ambiguity in computational models at the lexical level.\footnote{Other resources which specify relationships between lexical items (e.g., FrameNet) may also be used.}
WordNet groups synonymous words into synsets (i.e., concepts), with polysemous words belonging to different synsets. Synsets have multiple labeled relations among them, and in this paper, we focus on the hypernymy relation which defines a hierarchy of lexical generalizations.
For example, the word \textit{pool} belongs to nine different synsets (e.g., swimming pool, collection of things, pocket billiards).
\footnote{Unlike the examples listed here, the distinctions between some synsets in WordNet are very specific. For example, one of the nine synsets for \textit{pool} is defined as ``a communal combination of funds.''}
For example, the direct hypernym of the pocket billiards sense of \textit{pool} is \textit{table game}.
The hypernym paths are different for different senses of the word, with some hypernyms shared among senses.

We map each word type to a grid of concept embeddings (see Fig.~\ref{fig:ontolstm_tensor}), which are shared across many words. The word representation is computed as a distribution over the concept embeddings from the word's grid. We show how these distributions can be 
learned conditioned on the context when the representations are plugged into RNNs.
Intuitively, commonsense knowledge encoded as WordNet relations could potentially help with language understanding tasks. 
but mapping tokens to entities in WordNet is a challenge. One needs to at least disambiguate the sense of the token before being able to use the relevant information. Similarly, not all the hypernyms defined by WordNet may be useful for the task at hand as some may be too general to be informative.
 
\begin{figure}[t]
\begin{center}
\includegraphics[width=3in]{figures/tensor2.png}
\caption{Example of ontology-aware lexical representation}
\label{fig:ontolstm_tensor}
\end{center}
\end{figure}

\subsection{Related Work}
% concept embeddings
Previous work used lexical ontologies such as WordNet to improve \textit{pretrained} word embeddings at the \textit{type level}.
\cite{yu:14} extended the CBOW model \citep{mikolov:13} by adding an extra term in the training objective for generating words conditioned on similar words according to a lexicon.
\cite{jauhar:15} extended the skipgram model \citep{mikolov:13} by representing word senses as latent variables in the generation process, and used a structured prior based on the ontology.
\cite{faruqui:15} used belief propagation to update pretrained word embeddings on a graph that encodes lexical relationships in the ontology.
In contrast, we propose an approach for obtaining \textit{context-sensitive} embeddings at the token level, while \textit{jointly} optimizing the model parameters for the NLP task of interest.
Previous work used WordNet and other lexical resources such as the paraphrase database of \cite{ganitkevitch:13} to improve type-level word embeddings.
Related to the idea of concept embeddings is \cite{rothe:15} who estimated WordNet synset representations, given pretrained type-level word embeddings.
 In contrast, our work focuses on estimating token-level word embeddings given concept embeddings.
 Related to token-level embeddings is \cite{belanger:15} who proposed a Gaussian linear dynamical system for estimating token-level word embeddings. However, this approach does not make use of lexical ontologies and is not amenable for joint training with a downstream NLP task.

\subsection{Ontology-aware Recurrent Neural Networks}
\subsubsection{Lexical Representation}
\label{sec:ontolstm_input_rep}
To address the semantic ambiguity in traditional word embeddings, we represent each word type as a grid of concept embeddings, organized in a third-order tensor (i.e., a 3-dimensional array), as shown in Fig.~\ref{fig:ontolstm_tensor}, 
which is an example for the word \textit{pool}. The three dimensions correspond to senses, hypernyms per sense, and dimensionality of the synset embeddings respectively. We emphasize that synset embeddings have one-to-one correspondence with WordNet synsets,
and are shared across the tensor representation of multiple word types.

We use $T_w$ to denote the third-order tensor representation of a word $w$.
and $T_w(s,y)$ to refer to the $d$-dimensional concept embedding in the tensor which corresponds to a word sense $s$ and a hypernym $y$; e.g., $T_{\text{pool}}(\text{Pocket billiards},\text{Table game})$.
Other words share the same concept embedding, such as $T_{\text{dominoes}}(\text{dominoes game},\text{Table game}) = T_{\text{pool}}(\text{Pocket billiards},\text{Table game})$.
We truncate the tensor or pad it with zeros as needed, since different word types have different numbers of word senses, and different synsets have different numbers of hypernyms. 
With $S$ denoting a predefined, fixed number of senses per word type and $Y$ denoting a predefined, fixed number of generalizations (i.e., number of hypernyms plus one) per word sense, all tensor embeddings are of the same size: $T_w \in \mathbb{R}^{S\times Y \times d}$.
Unlike type-level representations in the proposed setup, rare words can leverage the concept embedding parameters shared with other words close to them in the WordNet graph structure.

\subsubsection{Ontology-Aware LSTM (OntoLSTM)}
\label{sec:ontolstm}
Recurrent Neural Networks (RNN) model sequences by repeating the same neural network module (i.e., the same model parameters) for each position in the sequence.
RNNs have been widely used for several NLP tasks, including language modeling \citep{mikolov:10}, textual entailment \citep{bowman:15}, speech recognition \citep{graves:13}, machine translation \citep{sutskever:14} and dependency parsing \citep{dyer:15}.
We now show how to use this lexical representation for modeling sequences by plugging them into long short-term memory networks  (LSTM, \cite{hochreiter1997long}) a popular type of RNNs.
At position $t$, the LSTM consumes the previous hidden state of the sequence $h_{t-1}$ and a vector representation of the input word at this position $i_t$, and produces the current hidden state $h_t$.\footnote{The interface of LSTM modules also includes a cell state which runs down the sequence with only minor linear transformations.}
Instead of using type-level word embeddings to represent the input at each position, we consider using token-level, ontology-aware word embeddings.
\begin{figure}
  \begin{center}
  \includegraphics[width=5in]{figures/ontolstm_diagram_modified.png}
  \caption{Schematic of OntoLSTM}
  \label{fig:ontolstm_model}
  \end{center}
 \end{figure}

The modification we propose is specific to the representation of input words ($i_t$). 
At each position $t$, we dynamically compute $i_t$ (corresponding to the token $w_t$) as a function of (a) the concept embeddings in $T_{w_t}$ and (b) the output of the previous timestep $h_{t-1}$, as shown in Figure~\ref{fig:ontolstm_model}.
Precisely, $i_t = \mathbb{E}_{p(s,y\mid w_t, h_{t-1})}\big[ T_{w_t}(s,y) \big]$ is the expected value of relevant concept embeddings, according to a discrete probability distribution over senses $s$ and hypernyms $y$.

\paragraph{Uniform attention:} The first variant of our model defines a uniform distribution for $p(s,y \mid w_t, h_{t-1})$. That is, we take a simple average of the synset embeddings in the grid to get the word representation. We call this variant \textbf{$\text{OntoLSTM}_{\text{uni}}$}.

\paragraph{Parameterized attention:}
The second variant of the model learns to weigh hypernyms differently conditioned on the context.
Although attention mechanisms are typically used to explicitly represent the importance of each item in a sequence \citep{bahdanau:14}, the can also be applied to non-sequential items.

We calculate an attention score for each generalization of each word sense in the following way:
\begin{equation}
\begin{aligned}[c]
M_t^y &= \tanh(T_{w_t}W_i^y + (\mathbbm{1}_{S \times Y}\otimes h_{t-1}) W_h^y) \nonumber \\
a_t^y &= \mathrm{softmax}(M_t^y q, 2) \nonumber \\
\end{aligned}
\begin{aligned}
&\in \mathbb{R}^{S\times Y \times l} \\
&\in \mathbb{R}^{S\times Y} \\
\end{aligned}
\end{equation}
where $(\mathbbm{1}_{S \times Y}\otimes h_{t-1}) \in \mathbb{R}^{S \times Y \times d}$ replicates the hidden state one time for each (word sense, generalization) pair then multiplies $W_h^y \in \mathbb{R}^{d \times l}$ producing $M_t^y \in \mathbb{R}^{S\times Y \times l}$. $W_i^y \in \mathbb{R}^{d \times l}$, $W_h^y \in \mathbb{R}^{d \times l}$ and scoring operator $q \in \mathbb{R}^l$ are model parameters. $\mathrm{softmax}(A,2)$ applies the softmax operation along the second dimension so that the attention scores of all generalizations of a word sense sum to one. Intuitively, $M_t^s$ computes a dense vector representation of length $l$ for each word sense, which is then multiplied by $p$ and normalized to compute the attention scores $a_t^s$. Similarly, $M_t^y$ operates on each (word sense, generalization) pair, and $q \in \mathbb{R}^l$ and gives hypernym attention $a_t^y$.

\paragraph{Sense Priors}: WordNet organizes senses of a word in the order of their frequency (i.e.the first sense of any word is its most frequent sense). We account for this property of the ontology by setting a prior on the sense probability. Precisely,
\begin{equation*}
    p(s|w_t) \sim \text{Exponential}(\lambda_{w_t})
\end{equation*}
We define one rate parameter $\lambda_{w_t}$ per each word type.

Finally, $i_t$ is computed as:
\begin{align}
%i_t &= \mathds{E}_{p(s,y\mid w_t, h_{t-1})}\big[ T_{w_t}(s,y) \big] \nonumber \\
i_t &= \sum_{s, y} p(s|w_t) \times a_t^y(s,y) \times T_{w_t}(s,y) \nonumber
\end{align}
We call this variant \textbf{$\text{OntoLSTM}_{\text{att}}$}.
%Note that the number of input representation parameters in both variants of OntoLSTM is not significantly higher than that of the LSTM. This is because the vocabulary sizes are comparable in both models, and tensors in OntoLSTM share many of the same synset representations.

\subsection{Experiments}
\label{sec:ontolstm_experiments}
We evaluate our model on the natural language inference task.
We show that $\text{OntoLSTM}_{\text{uni}}$ and $\text{OntoLSTM}_{\text{att}}$ improve the test set accuracy of the LSTM baseline by 1.9 and 1.8 absolute points, respectively.

\paragraph{Data preprocessing:} The datasets are part-of-speech (POS) tagged using Stanford's POS tagger \citep{toutanova:03}.
To construct the tensor representation, we map the POS-tagged word to the first two synsets in WordNet (i.e. $S=2$), and extract the most direct five hypernyms (i.e., $H=5$). When a synset has multiple hypernym paths, we use the shortest one. In preliminary experiments, we found that using more word senses or more hypernyms per word sense does not improve the performance.
Words which do not appear in WordNet are assumed to have one unique synset per word type with no hypernyms.

% \subsection{Language Modeling}
% \label{sec:lm}

% Here, we use a language modeling task to compare the three RNN models: LSTM, $\text{OntoLSTM}_\text{uni}$ and $\text{OntoLSTM}_\text{att}$.

% %\wacomment{People wanted to know what the SOTA in LM is. We should add citations to support comparing against LSTM. It would be nice if we can show that existing models have test PPL in the same ballpark as our baseline.}


% \paragraph{Setup:}
% All three models predict the next word in a sequence using a tree-factored softmax \cite{baltescu2014pragmatic}.
% The models are trained to maximize log-likelihood of training data, using stochastic gradient descent with early stopping (up to twenty epochs).
% We use train, development and test splits of sizes 18 million, 2 million and 2 million words, respectively, from the Associated Press newswire data in the 3rd edition of the English Gigaword. The vocabulary during training is initialized with words from training and test sets, so all three models have the same vocabulary with no out of vocabulary words.
% \pdcomment{This will change. Results with word and synset singletons in training data replaced with UNK coming soon.}
% All hidden layers, word embeddings and concept embeddings are of size 50.

% %Our models are capable of computing embeddings for words not seen in training by leveraging the underlying concept embeddings shared with other words seen in training.
% %To illustrate this important feature, we provide a breakdown of the perplexities for seen and unseen words (in training), using a fixed vocabulary which includes all words in the train, development and test data splits.
% %\wacomment{Chris, please review this paragraph.}

% \begin{table}
%     \centering
%     \begin{tabular}{|r|r|r|}
%     \hline
%     \textbf{Model} & Train PPL & Test PPL\\
%     \hline
%     LSTM & 258.8 & 264.0\\
%     $\text{OntoLSTM}_{\text{uni}}$ & 159.1 & 160.4\\
%     $\text{OntoLSTM}_{\text{att}}$ & 156.4 & 156.3\\
%     \hline
%     \end{tabular}
%     \caption{Language model perplexities (lower is better) of the train and test splits with three models. $\text{OntoLSTM}_{\text{uni}}$ and $\text{OntoLSTM}_{\text{att}}$ improve the test set perplexity of the LSTM baseline by 39.2\% and 40.7\%, respectively.}
%     \label{tab:lm_results}
% \end{table}

\begin{table}
    \centering
    \begin{tabular}{|l|c|c|c|}
    \hline
    \textbf{Model} & \textbf{Pretrained } & \textbf{Train} & \textbf{Test}\\
    \hline
    Our LSTM                        & No &  &  \\
    $\text{OntoLSTM}_{\text{uni}}$  & No &  &  \\
    $\text{OntoLSTM}_{\text{att}}$  & No &  &  \\ \hline
    LSTM & Yes &   &  \\
    \hline
    \end{tabular}
    \caption{Results on the SNLI dataset. The numbers are accuracies in percentages. Bottom row shows the results reported by \cite{bowman2016fast} using GloVe vectors to represent words.}
    \label{tab:ontolstm_snli_results}
\end{table}
% \paragraph{Results:}
% Table~\ref{tab:lm_results} reports perplexities (lower is better) of the train and test data splits using each of the three language models.
% The $\text{OntoLSTM}_\text{uni}$ model achieves 38\% and 39\% reduction in perplexities of the train and test data splits, respectively.
% %The $\text{OntoLSTM}_\text{uni}$ \wacomment{\ldots}

%\wacomment{volkan:I am not use but this dataset is not standard for LM testing. People usually use 1B word benchmark and PTB (although PTB is not big enough for LM testing). I would go for 1B word benchmark since there are plenty of results on that dataset.}
%\wacomment{volkan: it is important to see the effect of word embedding. I would love to see i) fixed embedding training with glove initialization ii) learning embeddings from scratch (what you reported) iii) finetuning glove with your model. I bet your code is in theano/keras. I can show you how to do finetuning if you need any help. waleed: honestly, i don't think this actually important given the time constraints.}

% \subsection{Natural Language Inference}
% \label{sec:textual_entailment}

%\wacomment{volkan:table-2 should definitely have sota results from bowman et. al. 2006 in bottom row}

% Here, we use a natural language inference task (also known as recognizing textual entailment) to compare the three RNN-based models: LSTM, $\text{LSTM}_\text{uni}$ and $\text{LSTM}_\text{att}$.

\paragraph{Task:}
Given a pair of sentences (premise and hypothesis), the model predicts one of three possible relationships between the two sentences: \textit{entailment}, \textit{contradiction} or \textit{neutral}.
We use the standard train, development and test splits of the SNLI corpus, which consists of 549K, 10K and 10K labeled sentence pairs, respectively.
For example, the following sentence pair is labeled \textit{entailment}:
 \begin{itemize}
  \item \textbf{Premise} \textit{Children and parents are splashing water at the pool.}
  \item \textbf{Hypothesis} \textit{Families are playing outside.}
 \end{itemize}
\paragraph{Model:}
The three models we compare use the same architecture, except for the representation of input tokens.
Following \cite{mou2015recognizing}, we use two LSTM with tied parameters to read the premise and the hypothesis of each example, then compute the vector $h$ which summarizes the sentence pair as the concatenation:
$$h=\big[h_{\text{pre}}; h_{\text{hyp}}; h_{\text{pre}} -h_{\text{hyp}};h_{\text{pre}} *h_{\text{hyp}}\big]$$
where $h_\text{pre}$ is the output hidden layer at the end of the premise token sequence, $h_\text{hyp}$ is the output hidden layer at the end of the hypothesis token sequence.
The summary $h$ then feeds into two fully connected ReLU layers of size 1024, followed by a softmax layer of size 3 to predict the label.
The word embeddings and concept embeddings are of size 300, and the hidden layers 150.
The models are trained to maximize log-likelihood of correct labels in the training data, using ADAM \citep{kingma2014adam} with early stopping (up to twenty epochs).

\paragraph{Results:}
Table~\ref{tab:ontolstm_snli_results} shows the classification results. We also report previously published results of a similar neural architecture as an extra baseline: the 300-dimensional LSTM RNN encoder model in \cite{bowman2016fast}.\footnote{We note this is not the best model presented in \cite{bowman2016fast}, but it is the one that is most similar to our neural architecture.}
They use GloVe as pretrained word embeddings while we jointly learn word/concept embeddings in the same model.
\cite{bowman2016fast}'s LSTM outperforms our LSTM model by 1.5 absolute points on the test set. This may be due to the difference in input representations. Since we learn the synset representations in OntoLSTM, comparison with our LSTM implementations is more sensible.
The $\text{OntoLSTM}_{\text{uni}}$ outperforms the LSTM model with 1.8 absolute points on the test set, illustrating the utility of the ontology-aware word representations in a controlled experiment.

\subsection{Analysis}
\label{sec:discussion}
\begin{figure}
\begin{center}
\includegraphics[width=5in]{figures/ontolstm_snli_comparison.png}
\caption{Relative attention values for words in the premise and in the hypothesis.}
\label{fig:snli_visualization}
\end{center}
\end{figure}

\paragraph{Generalization attention:} \label{sec:generalization}
Fig.~\ref{fig:snli_visualization} shows one example from the test set of the SNLI dataset where hypernymy information is useful. 
The LSTM model shares no parameters between the lexical representations of \textit{book} and \textit{manuals}, and fails to predict the correct relationship (i.e., entailment). 
However, $\text{OntoLSTM}_{\text{att}}$ leverages the fact that \textit{book} and \textit{manuals} share the common hypernyms \textit{book.n.01} and \textit{publication.n.01}, and assigns relatively high attention probabilities to these hypernyms, resulting in a similar lexical representation of the two words, and a correct prediction of this example.

The following is an example that LSTM gets right and OntoLSTM does not:
\begin{itemize}
 \item Premise: \textit{Women of India performing with blue streamers, in beautiful blue costumes.}
 \item Hypothesis: \textit{Indian women perform together in gorgeous costumes}
\end{itemize}
\textit{Indian} in the second sentence is an adjective, and \textit{India} in the first is an noun. By design, WordNet does not link words of different parts of speech. Moreover, the adjective hierarchies in WordNet are very shallow and \textit{gorgeous} and \textit{beautiful} belong to two different synsets, both without any hypernyms. Hence OntoLSTM could not use any generalization information in this problem.

\paragraph{Parameter space:} We note that the model size in LSTM and both variants of OntoLSTM is comparable (11.1M and 12.7M parameters, respectively). This is because the synset representations are shared across words, rendering the vocabulary sizes in OntoLSTM camparable to those in LSTM. We have about 1.7M parameters for the remaining parts of the models in both cases. 
%\newcite{bowman2016fast} report that they have 3M parameters in their model, none for the input representations themselves.

\section{Proposed Work: Augmenting Proposition Encoding using Graph Features}
As an extension to the implemented ideas above, we propose encoding propositions guided by subgraph features from a knowledge base. Concretely, the inputs that need to be encoded are nested propositions parsed using an open-vocabulary semantic parser. The reason we use an open vocabulary
semantic parser 
\begin{figure}
\begin{center}
\includegraphics[width=5in]{figures/knowledge_backed_prop_tree.png}
\caption{Knowledge backed proposition tree}
\label{fig:kb_prop_tree}
\end{center}
\end{figure}
\chapter{Reasoning Using Knowledge: Memory Networks for Deep Reasoning}
\label{chapter:memnet_qa}
\section{Memory Networks}
%TODO: General introduction about the need for explicit memory.
Memory Networks (MemNet) are a class of models
that combine inference with long-term memory. Unlike Recurrent Neural Networks
(RNN) that model language \citep{mikolov2010recurrent}, and their variants with
Long Short-Term Memory (LSTM) \citep{hochreiter1997long}, MemNets have an explicit
memory component with read and write functions. While the original MemNet model 
proposed by \cite{weston2014memory}, MemNN required explicit supervision
for selecting the relevant parts of the memory, \citep{sukhbaatar2015end}
proposed a end-to-end variant (MemN2N) where the memory selection component is
trained jointly with the rest of the network. These were previously used for
answering questions that require reasoning over multiple context sentences,
both in simulated \citep{bordes2010towards} and large-scale
\citep{fader2013paraphrase} scenarios.

In this chapter, we use the term \textit{memory network},
or the abbreviation \textit{MemNet}, to refer to any neural network model with an explicit memory component
that can be read from or written to. Our focus is mostly on the general class of models. Wherever necessary,
we refer to the original memory network model
proposed by \cite{weston2014memory} as \textit{MemNN} and the end to end model by \cite{sukhbaatar2015end}
as MemN2N.


\begin{figure*}
\begin{center}
  \includegraphics[width=6.5in]{figures/memory_network_generic.png}
  \caption{Schematic showing a generic view of end-to-end memory network}
  \label{fig:memnet}
  \end{center}
\end{figure*}
Figure~\ref{fig:memnet} shows the setup of a generic MemNet. It takes as
input a set of $N$ context sentences indexed as $\{c_i\}_{i=1}^N$, such that
$c_i$ is a vector containing the indices of words in the $i^\text{th}$ sentence.
The sentences are then encoded using the \texttt{EncodeContext} function to
produce the matrix $C \in \mathbb{R}^{n \times d}$, where each row $C[i]$ is the
encoding of sentence $c_i$. In addition, the model also takes as input a query
indexed as $q$, which is a vector containing the indices of the words in the
query similar to $c_i$ vectors. The query is encoded using \texttt{EncodeQuery}
to produce $u^0 \in \mathbb{R}^d$. The memory network can have multiple memory
layers, corresponding to multiple hops. At each hop, a memory layer receives as
input the output from the previous hop $u^{h-1}$, which is passed to the
\texttt{SelectMemory} function, which uses an attention mechanism to select the
relevant parts of the encoded context, conditioned on $u^{h-1}$ and produces
a summary $s^h$, of the context encoding for the current hop. $s^h$ is then
passed to \texttt{UpdateMemory} along with $u^{h-1}$, to produce the updated
memory representation $u^h$ for the current hop. It has to be noted that the
initial $u^0$ is the encoding of the query itself. Finally, an answer is
predicted by passing the query encoding $u^0$, and the summary of the
context, $s^H$ from the final hop $H$ to the \texttt{PredictAnswer} function.

It can be seen that MemN2N fits into this setup with the following
configuration:
% TODO(pradeep): MemN2N actually has two Embedding matrices encoding background
% for input and output. Also, there is another variant which encodes positions of
% the words too.
\begin{flalign*}
&\texttt{EncodeQuery}(q) = \text{Embedding}_q(q) \\
&\texttt{EncodeContext}(c_i) = \text{Embedding}_c(c_i) \\
&\texttt{SelectMemory}(u^{h-1}, C) = \text{softmax}(C.u^{h-1}).C \\
&\texttt{UpdateMemory}(u^{h-1}, s^h) = u^{h-1} + s^h \\
&\texttt{PredictAnswer}(u_0, s^H) = \text{softmax}(W.s^H)
\end{flalign*}
where $\text{Embedding}_q(.)$ and $\text{Embedding}_c(.)$ are simply bag of
words models that aggregate the vector representations of all the words given by
the indices in the input. $W \in \mathbb{R}^{V \times d}$ is a parameter of the
answer prediction function, causing the softmax to be over the vocabulary size
$V$. Note that in MemN2N, $u^0$ is not an argument of \texttt{PredictAnswer}.

In this chapter, we focus on problems where the context is not well-defined, and needs to
be retrieved for a big corpus. Accordingly, we propose to add an additional
\texttt{RetrieveContext} module to our generic MemNet architecture.
Particularly, we are interested in developing MemNet models that
can figure out whether the available context is sufficient to make a prediction. We first
describe a target dataset for our model, and then list the issues we propose to address.

\section{Science Question Answering}
The ScienceQA dataset we use here is significantly different from the
QA datasets previously used to test memory networks, and requires more complex
reasoning. One example of such question is shown below.
%TODO(pradeep): Make this a table?
\begin{itemize}
\item Astronauts weigh more on Earth than they do on the moon because
\begin{enumerate}[(a)]
 \item they have less mass on the moon
 \item their density decreases on the moon
 \item the moon has less gravity than Earth
 \item the moon has less friction than Earth
\end{enumerate}
\end{itemize}
%TODO(pradeep): Say more things about the nature of the problem.
The text relevant to questions like this may contain a single sentence that
has the information to answer this question, like this:
\begin{itemize}
 \item People weigh more on some planets than others because of differences in gravity.
\end{itemize}

Or it may span multiple sentences like this:
\begin{itemize}
 \item Moon's gravity is less than that of Earth.
 \item Weight of a person depends on the gravity of the planet.
\end{itemize}

Or in other cases, the text may not contain the relevant answer at all. Hence, this setup
requires an module in addition to the ones mentioned that retrieves relevant content from
the corpus. However, as the size of the context increases, it becomes more and more expensive to reason over
it. Also, increased context size may add noise to prediction process. Hence, it is important to
retrieve context conservatively, and the model should reason whether the retrieved context
is sufficient to produce an answer.

\paragraph{Question Answering as Textual Entailment} In our proposed setup, we
cast the problem of deciding whether the given context is sufficient to answer the question, as a
textual entailment problem. That is, given a multiple choice question with answer options, we
convert the combination of the question and each of the options into a
statement, and check whether the statement can be entailed from relevant
background information. Given this setup, the \texttt{PredictAnswer} function
essentially becomes an entailment function. If none of the question-option combinations
can be entailed from (or contradicted by) the retrieved context, we retrieve more context.

\subsection{Proposed Algorithm}
Based on the ideas described so far, we now present the proposed algorithm, \textsc{AdaptiveRead}
to read and reason, while choosing to retrieve or predict depending on the sufficiency
of the context available.

\begin{algorithm}[H]
 \KwIn{$L$: large corpus, $q$: question}
 \KwOut{$a$: answer}
 $ContextSize = k$ \;
 $C = \texttt{RetrieveContext}(q, L, ContextSize)$, the top $k$ relevant sentences from the corpus\;
 \While{$ContextSize < MaxContextSize$} {
  \eIf{$\texttt{EntailsOrContradicts}(C, q)$}{
    $a = \texttt{PredictAnswer}(C, q)$ \;
    return $a$
  }{
    $ContextSize += k$ \;
    $C = \texttt{RetrieveContext}(q, L, ContextSize)$ \;
  }
 }
 \caption{\textsc{AdaptiveRead} algorithm that learns when to stop retrieving context}
\end{algorithm}

This algorithm shares some similarities with the \textit{ReasoNet} model
proposed by \cite{shen2016reasonet}, but the main difference is that while ReasoNet learns to
stop performing additional hops, the proposed algorithm learns to stop retrieving additional context.

\subsection{Potential Issues}
We identify the following potential issues in building \textsc{AdaptiveRead}.
\begin{enumerate}
 \item \textbf{Scalability to large corpora}: Depending on the complexity of \texttt{PredictAnswer}, reasoning
 over large corpora may prove to be intractable. Earlier work in solving this problem involves building
 hierarchical models \cite{chandar2016hierarchical,choi2016hierarchical}. While \cite{chandar2016hierarchical}
 build a hierarchical memory network that uses a simple dot product for \texttt{RetrieveContext} and \texttt{SelectMemory},
 and thus use maximum inner product search (MIPS) to do them efficiently, \cite{choi2016hierarchical} perform summarization
 to retrieve context followed by answer prediction. We can use some of their ideas to make \texttt{RetrieveContext} a simpler
 process, and \texttt{SelectMemory} a more expensive one.
 
 \item \textbf{Difficulty in training}: Clearly, \textsc{AdaptiveRead} makes a hard choice between retrieving context and
 predicting an answer, thus making it impossible to train the model end-to-end using back-propagation. \cite{shen2016reasonet}
 and \cite{choi2016hierarchical} get around this problem using REINFORCE \cite{williams1992simple} algorithm.
\end{enumerate}


% \subsection{Preliminary Results}
% We performed some preliminary experiments to understand the 
% We now show some preliminary results of our memory network implementation on science question answering using MemN2N architecture with following change in configuration:
% \begin{flalign*}
% &\texttt{PredictAnswer}(u_0, s^H) = \texttt{HeuristicMatch}(u_0, s_h)
% \end{flalign*}
% where \texttt{HeuristicMatch}(.) is the heuristic matching function proposed by \cite{mou2015recognizing} for textual entailment, also used for our experiments with SNLI data in 
% Chapter~\ref{chapter:ontolstm}. The dataset used for these experiments is questions collected from 4th and 8th grade science text books.
% Context for each question was obtained as follows. We built a Lucene index over a big collection of sentences related to general science from various sources and extracted relevant background sentences for each question by querying it. For each of the
% options, we query the Lucene index with a combination of the option text and the question. We thus transform questions like the one shown above into four entailment problems where the combination of the
% question and one of the options is the hypothesis and the relevant background sentences are the premises. The final answer is picked by selecting the option (combined with the question) that has the highest
% entailment score. This is done by passing the final entailment scores through a softmax layer.
% 
% Preliminary results using our model are shown in Table~\ref{tab:memnet_qa_results}, using two different encoders
% to encode the input sentences and background. We also show the accuracy of a baseline system based on Lucene, that selects for every question, the option that results in the highest relevance score in the process
% described above for retrieving background sentences. It can be seen that the simple baseline does better than the memory network model.
% 
% \begin{table}
%     \centering
%     \begin{tabular}{|l|c|}
%     \hline
%     \textbf{Encoder} & \textbf{Test Acc.}\\
%     \hline
%     BOW & \% \\
%     GRU & 38.8\% \\
%     \hline
%     \hline
%     \textbf{Lucene baseline} & 41.1\% \\
%     \hline
%     \end{tabular}
%     \caption{Results of our Memory Network on ScienceQA in comparison with a Lucene baseline}
%     \label{tab:memnet_qa_results}
% \end{table}
% 
% \subsection{Analysis}
% 
% \subsection{Proposed Work}



\chapter{Non-standard NLU Tasks}
\label{chapter:other_tasks}
\input{chapters/other_tasks}
\chapter{Transfer Learning}
\label{chapter:transfer_learning}
So far we have evaluated our NLU pipelines by measuring the accuracy of the prediction component at any given task. The encoder component has not been explicitly evaluated.
This is because we generally view NLU pipelines as end-to-end systems, and do not care about the quality of the encoding as long as it serves as good features for the prediction
component. In this chapter, we take a different view and try to evaluate the encoder by investigating how well the encoded representations can be transferred across tasks. Transfer
learning in NLU systems is an attractive line of investigation for various reasons:
\begin{itemize}
 \item Not all language understanding tasks come with lots of training data. For some tasks it may be very difficult to get high quality annotations from humans due to factors such as 
 the task being subjective or requiring domain expertise. Being able to pre-train encoders on one task that has large amounts of training data available (like SNLI \cite{bowman:15} for textual entailment), and use them for similar
 tasks like paraphrasing or question answering, can address this issue.
 \item From a language understanding perspective, the transferability of the encoders gives us a clearer idea of what kind of features they are learning.
 \item Due to the large number of parameters involved, deep neural networks are highly prone to overfitting to the task. Transfer learning can address this problem. 
\end{itemize}

Transfer learning in neural networks has been well-studied for vision related problems. It has been shown \citep{zeiler2014visualizing} that object recognition models trained on
ImageNet \citep{deng2009imagenet} can be transfered to other object recognition datasets when the final classifier is retrained to the new dataset. A systematic study of the generality
of features learned at various layers in deep neural networks trained on natural images was done by \cite{yosinski2014transferable}. For language related problems, there do not exist similar studies.
There are however multi-task learning efforts for NLP tasks.  A notable example is the work by \cite{collobert2011natural} that showed the benefit of multi-task learning for surface-level NLP tasks 
like Named Entity Recognition, Part of Speech Tagging and Semantic Role Labeling. In this chapter, we investigate the transferability of the encoder in an NLU system trained on one task, to other similar tasks.

\section{Proposed Work}
The proposed work in this area involves answering the following concrete questions.
\begin{enumerate}
 \item Can different NLU tasks be jointly learned using the same neural network architecture?
 \item In a scenario where the NLU system is pretrained on a task with large amounts of data, how does the complexity (in terms of number of parameters)
 of the encoder affect its transferability to other tasks?
 \item How do external knowledge inputs affect the transferability of the encoder?
\end{enumerate}

\chapter{Summary and Timeline}
\section{Summary}
In this thesis we propose to augment deep learning systems for NLU with external knowledge inputs. We identify two broad kinds of knowledge: \textit{background} and \textit{contextual}.

We defined background knowledge as the implicit shared human knowledge that is often omitted in human generated text or speech. Our proposal to incorporate this kind of knowledge in NLU systems is by linking text to structured knowledge bases like
WordNet, Freebase and ConceptNet, and consequently modify the encoder to process this additional information. In Chapter~\ref{chapter:ontolstm}, we shouwed the advantages of linking words to WordNet subgraphs, and how it helps textual entailment and 
prepositional phrase attachment tasks. By giving the encoder, access to the hypernym paths of relevant concepts we showed that we can learn useful task-dependent generalizations using an attention mechanism. We propose to use similar disambiguation
techniques to learn better representations of nested proposition structures by linking the nodes of a TreeLSTM to subgraph structures from Freebase, and better event representations by linking semantic role labeled structures with commonsense information from ConceptNet.

We defined contextual knowledge as the explicit additional information that reading comprehension systems need to reason over, during prediction. We identified that many reading comprehension tasks require modeling complex entailment,
which is beyond the capability of currently published memory networks. Our proposal in Chapter~\ref{chapter:memnet_qa} includes modifying the prediction component of memory networks to make the applicable to science question answering and modeling experiment narratives.

\section{Timeline}
\begin{itemize}
    \item Ontology Aware Recurrent Neural Networks: Done
    \begin{itemize}
        \item Target: ACL 2017
    \end{itemize}
    \item Complex entailment in memory networks for Science QA: Nov 2016 -- Feb 2017
    \begin{itemize}
        \item Target: ACL 2017
    \end{itemize}
    \item KB TreeLSTM for question answering: Feb 2017 -- May 2017
    \begin{itemize}
        \item Target: NIPS 2017 or EMNLP 2017
    \end{itemize}
    \item Improved event understanding with ConceptNet: May 2017 -- August 2017
    \item Transfer Learning with background knowledge: August 2017 -- November 2017
    \begin{itemize}
        \item Target: ICLR 2018
    \end{itemize}
    \item Thesis writing and defense: Spring 2018
\end{itemize}


%\appendix
%\include{appendix}

\backmatter

%\renewcommand{\baselinestretch}{1.0}\normalsize

% By default \bibsection is \chapter*, but we really want this to show
% up in the table of contents and pdf bookmarks.
\renewcommand{\bibsection}{\chapter{\bibname}}
%\newcommand{\bibpreamble}{This text goes between the ``Bibliography''
%  header and the actual list of references}
\bibliographystyle{plainnat}
\bibliography{thesis} %your bib file

\end{document}
